\documentclass[a5paper,10pt,twoside,openany,french]{book}

\usepackage{corbac} % Utilisation du paquet spécifique à Liam


\author{Liam \textsc{Uguel}}
\title{La Plume et la Griffe}
\date{2022}


\renewcommand\seal{
  \Large
  \noindent\begin{TAB}(e,0.4cm,0.4cm){cc}{cc}
    掠&夺\\ % Corbeau
    愿&舵 % Volonté du heame
  \end{TAB}
}

% Comenter pour désactiver l’ajout automatique des sceaux dans les illustrations.
\addseal


\begin{document}
  \frontmatter
  \maketitle

  \mainmatter


%  MODÈLE À SUIVRE :
%
% \begin{liamchapter}{<Titre du chapitre>}{</chemin/vers/l’image.png>}
%
% <Bla bla principal du paragraphe>
%
% \end{liamchapter}

  \begin{liamchapter}{Le vent dans la plume}{example-image-a}
  \lettrine[lines=3]{P}{our vous faire} mieux connaitre d’où vient l’erreur de ceux qui blâment la volupté, et qui louent en quelque sorte la douleur, je vais entrer dans une explication plus étendue, et vous faire voir tout ce qui a été dit là-dessus par l’inventeur de la vérité, et, pour ainsi dire, par l’architecte de la vie heureuse.

  Personne [dit Épicure] ne craint ni ne fuit la volupté en tant que volupté, mais en tant qu’elle attire de grandes douleurs à ceux qui ne savent pas en faire un usage modéré et raisonnable ; et personne n’aime ni ne recherche la douleur comme douleur, mais parce qu’il arrive quelquefois que, par le travail et par la peine, on parvienne à jouir d’une grande volupté. En effet, pour descendre jusqu’aux petites choses, qui de vous ne fait point quelque exercice pénible pour en retirer quelque sorte d’utilité ? Et qui pourrait justement blâmer, ou celui qui rechercherait une volupté qui ne pourrait être suivie de rien de fâcheux, ou celui qui éviterait une douleur dont il ne pourrait espérer aucun plaisir.

  Au contraire, nous blâmons avec raison et nous croyons dignes de mépris et de haine ceux qui, se laissant corrompre par les attraits d’une volupté présente, ne prévoient pas à combien de maux et de chagrins une passion aveugle les peut exposer.
  \end{liamchapter}

  \begin{liamchapter}{Quelques béquetées}{example-image-b}

    \lettrine[lines=3]{J’}{en dis autant} de ceux qui, par mollesse d’esprit, c’est-à-dire par la crainte de la peine et de la douleur, manquent aux devoirs de la vie. Et il est très facile de rendre raison de ce que j’avance. Car, lorsque nous sommes tout à fait libres, et que rien ne nous empêche de faire ce qui peut nous donner le plus de plaisir, nous pouvons nous livrer entièrement à la volupté et chasser toute sorte de douleur ; mais, dans les temps destinés aux devoirs de la société ou à la nécessité des affaires, souvent il faut faire divorce avec la volupté, et ne se point refuser à la peine.

    La règle que suit en cela un homme sage, c’est de renoncer à de légères voluptés pour en avoir de plus grandes, et de savoir supporter des douleurs légères pour en éviter de plus fâcheuses.
  \end{liamchapter}






  \backmatter

\end{document}
